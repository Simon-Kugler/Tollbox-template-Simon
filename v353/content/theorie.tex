\section{Zielsetzung}
Ziel des Versuches ist es, das Relaxationsverhalten eines RC-Kreises zu untersuchen.

\section{Theorie}
\label{sec:Theorie}
\subsection{Die allgemeine Relaxationsgleichung für den RC-Kreis}
Wird ein System aus dem Ausganszustand herausgebracht um dann 
nicht-oszillatorisch zurückzukehren, so spricht man von Relaxationserscheinungen.
Für die Änderungsrate der physikalischen Größe $A$ gilt 
\begin{equation*}
    \frac{\text{d}A}{\text{d}t}=c[A(t)-A(\infty)]\,.
\end{equation*}
Über Integration der Größe über die Zeit ergibt sich 
\begin{equation*}
    A(t)=A(\infty)+[A(0)-A(\infty)]e^{ct}\,.
\end{equation*}
Um $A$ beschränkt zu lassen, muss gelten dass $c<0$\,.
Als Beispiel wird nun das Auf- und Entladen eines Kondensators
mit Kapazität $C$ und Ladung $Q$ betrachtet.
Für die Entladung gilt für die anliegende Spannung 
\begin{align*}
    U_\text{C}&=\frac{Q}{C}\\
    \intertext{sowie für die Stromstärke}
    I&=\frac{U_\text{C}}{R}\,.
\end{align*}
Dabei ist $R$ der Widerstand, welcher einen Ladungsausgleich bewirkt.
Mit Lösen der Differentialgleichung 
\begin{align*}
    \frac{\text{d}Q}{\text{d}t}&=-\frac{1}{RC}Q(t)\\
    \intertext{ergibt sich als Funktion für $Q$}
    Q(t)&=Q(0)\exp\left(\frac{-t}{RC}\right)\,.
\end{align*}
Für unendliche Zeiten konvergiert $Q(t)$ gegen null.
Der Aufladevorgang wird, hergeleitet über einen ähnlichen Weg,
mit der Gleichung 
\begin{equation*}
    Q(t)=CU_0(1-\exp\left(\frac{-t}{RC}\right))
\end{equation*}
beschrieben.
Als Randbedingungen gelten hier $Q(0)=0$ sowie $Q(\infty)=CU_0$.
Die Zeitkonstante $RC$ ist ein Maß für jene Geschwindigkeit, mit der 
das System gegen den Grenzwert für $Q(t)\text{lim}t\rightarrow\infty$
konvergiert. Während der Zeit $\increment T$ ändert sich die Ladung um 
den Faktor $\exp(\sfrac{-(t=RC)}{RC})=\exp(-1)=e^{-1}\approx0,368$\,. 
\subsection{Relaxationsphänomene durch Periodizizät im Auslenkungsvorgang 
aus der Ruhelage}
Um weitere Analogien zu mechanischen phsyikalischen Systemen zu betrachten,
wird nun eine Sinusspannung betrachtet für welche man in der mechanischen 
Physik Sinusförmige Kräfte als Beispiel betrachten kann. Es gilt
\begin{equation*}
    U(t)=\cos(\omega t)
\end{equation*}
Dabei ist $\omega$ die Kreisfrequenz, bei der unter der Bedingung $\omega<<\sfrac{1}{RC}$
gilt, dass die Spannung $U_\text{C}$ am Kondensator genähert gleich mit der Erregerspannung ist
sowie die Phasenverschiebung $\varphi$ zwischen beiden Spannungen null ist.
Bei steigender Frequenz $\omega$ wird jedoch die Phasenverschiebung immer größer,
da die Auf- und Entladung des Kondensators länger andauert als das Überwinden des
Widerstands $R$. Ebenfalls sinkt die Spannung $U_\text{C}(t)$ am Kondensator.
Mathematisch ausgedrückt ist dieses Problem 
\begin{equation*}
    U_\text{C}(t)=A(\omega)\cos(\omega t+\varphi\{\omega\})\,.
\end{equation*}
Mit den Kirchhoffschen Gesetzen, einsetzen und Umformungen ergibt sich für die Phasenverschiebung
in Abhängigkeit der Frequenz
\begin{equation}
    \varphi(\omega)=\arctan(-\omega RC)\,.
    \label{eq:arctan1}
\end{equation}
Das Grenzverhalten des $\arctan$ für niedrige Frequenzen geht gegen null.
Werden sehr hohe Frequenzen eingestellt, konvergiert er gegen $\sfrac{\symup{\pi}}{2}$.
Für die Amplitude in Abhängigkeit der Frequenz kann der Ausdruck
\begin{equation}
    A(\omega)=\frac{U_0}{\sqrt{1+\omega^2R^2C^2}}
    \label{eq:arctan}
\end{equation}
hergeleitet werden. Für sehr kleine Frequenzen gilt $A(\omega)\approx U_0$, bei sehr großen
konvergiert $A(\omega)$ gegen null. Dieses freuqenzabhängige Verhalten ist das eines
Tiefpasses.
\subsection{Integration mithilfe des RC-Kreises}
Wenn gilt, dass $\omega >>\sfrac{1}{RC}$ ist, kann die Beziehung
$U_\text{C}\propto\int U(t)\text{d}t$ gezeigt werden. Ausgegangen wird von
$U(t)=U_\text{R}(t)+U_\text{C}(t)=I(t)\cdot R+U_\text{C}(t)$.
Es gelten die Vorraussetzungen
\begin{align*}
    \omega &>>\frac{1}{RC}\,,\\
    |U_\text{C}|&<<|U_\text{R}|\text{ sowie}\\
    |U_\text{C}|&<<|U|\,.
\end{align*}
Daraus lässt sich der Ausdruck
\begin{equation}
    U_\text{C}(t)=\frac{1}{RC}\int_0^t U(t')\text{d}t'
    \label{eq:integrator}
\end{equation}
herleiten.