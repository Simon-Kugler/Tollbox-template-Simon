\section{Diskussion}
\label{sec:Diskussion}
% Punkte für Diskussion 
% -Knopf unendlich durchdrehbar für feineinstellung der zeitteilung
% -Schwingungen waren nicht stationär 
% -Messwert bei Entladekurve rausgelassen (Kurve nicht genullt auf Oszi)
\subsection{Bestimmung der zeitkonstante bei konstanter Frequenz}
Zur ersten Bestimmung der Zeitkonstante wurden die Werte einer Entladekurve abgenommen.
Diese war zum Zeitpunkt der Messung jedoch so verschoben, dass ihr Anfang nicht bei $t=0$
war, sowie das Ende nicht bei $U=0$\,. Dies hat das Messen erschwert. Der letzte Messwert
wurde bewusst aus der Berechnung der Zeitkonstante herausgelassen, da dieser offensichtlich 
deutlich von einer Geraden durch die linearisierten Werte abweicht. Damit würde dieser einerseits
den Wert für $RC$ verfälschen, andererseits aber auch den Fehler der Zeitkonstante deutlich
erhöhen. $RC$ wurde damit als $\qty{0,52(0.04)}{\milli\second}$ berechnet, wobei der Fehler,
welchen SciPy berechnet hat relativ klein ist. Darin enthalten ist jedoch nicht der Fehler, der 
durch das erschwerte Ablesen aufgetreten ist. 
\subsection{Bestimmung der Zeitkonstante bei variierter Frequenz}
In diesem Abschnitt sollte die Zeitkonstante mithilfe zwei unterschiedlicher Mehtoden bestimmt werden.
Dazu wurde eine Messreihe durchgeführt, bei der zu sich steigernder Frequenz die Amplituden beier Kurven,
die Abstände ihrer Nulldurchgänge sowie die Periodenlönge gemessen wurde. Die Messung der Periodenlänge 
hat sich als nicht richtig herausgestellt, weswegen diese über $T=\sfrac{1}{f}$ berechnet wurde.
Wie schon in der Auswertung erwähnt, war die Feineinstellung der Zeiteinstellung des Oszilloskop am 
Vollausschlag nicht mehr arretiert, was zu einer völlig fehleingestellten Skala der Zeit führte. Dadurch 
konnte weder die Periodenlänge $(b)$, noch der Nulldurchgangsabstand $(a)$ sicher abgelesen werden. 
Außerdem konnten die Schwingungen nur sehr erschwert still auf dem Oszilloskop gehalten werden, was
ebenfalls einen großen, nicht skalierbaren Fehler mit sich brachte. 
Über die zweite Methode, das Auswerten des Amplitudenverhältnisses beider Schwingungen, wurde eine 
Zeitkonstante von $\qty{2,27(0.05)}{\milli\second}$ berechnet. Das entspricht einer Abweichung von
$77\,\%$ zwischen beiden errechneten Werten. Über die dritte Methode war eine Berechnung der Zeitkonstante 
nicht möglich, da die Messwerte (zu sehen in \autoref{fig:plotc}) keinerlei Parallelen zum Verlauf der 
Regressionskurve zeigten. 
Mit steigenden Frequenzen wurde das arretieren der Kurven auf dem Oszilloskop praktisch unmöglich, weshalb
der zur Messung verwendete Frequenzbereich deutlich früher endete, als eigentlich angedacht.
\subsection{RC-Kreis als Integrator}
Die visuelle Integration der verschiedenen Schwingungsformen erfolgte so wie erwartet und korrekt. Die 
Fotoqualität ist nicht optimal, da die Belichtungszeit der Kamera kein Foto vom Bildschirm zuließ.
Anstattdessen wurde ein Video aufgenommen und daraus ein Foto herausgeschnitten.  