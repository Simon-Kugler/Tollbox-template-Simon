\documentclass[titlepage=firstiscover, parskip=half, bibliography=totoc,captions=tableheading]{scrartcl} % KOMA-Script Dokumentenklasse Article

% Warnung, falls noch einmal kompiliert werden muss
\usepackage[aux]{rerunfilecheck}

% Paket für Schriftarteinstellung, muss immer geladen werden
\usepackage{fontspec}

% Deutsche Spracheinstellungen, wichtig z. B. für korrekte Trennung
\usepackage[ngerman]{babel}

% mehr Pakete hier
\usepackage{amsmath} % unverzichtbare Mathe-Befehle
\usepackage{amssymb} % viele Mathe-Symbole
\usepackage{mathtools} % Erweiterungen für amsmath
\usepackage[
math-style=ISO, % ┐
bold-style=ISO, % │
sans-style=italic, % │ ISO-Standard folgen
nabla=upright, % │
partial=upright, % │
mathrm=sym, % ┘
]{unicode-math}
\usepackage[
version=4,
math-greek=default,
text-greek=default,
]{mhchem}
\usepackage[section, below]{placeins}
\usepackage{caption} % Captions schöner machen
\usepackage{graphicx}
\usepackage{graphicx}
\usepackage{tabularray}
\UseTblrLibrary{booktabs}
\UseTblrLibrary{siunitx} % Lädt siunitx und definiert die S-Spalte
\usepackage[style=alphabetic]{biblatex} % nach babel
\addbibresource{lit.bib}

% Unterstützung für Links und PDF Metadaten
\usepackage[unicode]{hyperref}
\usepackage{bookmark}
\usepackage{microtype}
\usepackage{xfrac}
% Einstellungen hier, z.B. Fonts
\setmathfont{Latin Modern Math}
% \setmathfont{Tex Gyre Pagella Math} % alternativ




\begin{document}
\title{Abgabe 2}
\author{
Sophia Brechmann \\
\texorpdfstring{\href{mailto:sophia.brechmann@tu-dortmund.de}{sophia.brechmann@tu-dortmund.de}\and}{,}
Simon Kugler \\
\texorpdfstring{\href{mailto:simon.kugler@tu-dortmund.de}{simon.kugler@tu-dortmund.de}}{}
}
\date{Deadline: Dienstag, 31.10.2023}
\maketitle
\section{}
    \subsection{Was bezeichnet der Mittelwert?} Der Mittelwert beschreibt den durchschnittlichen Wert einer Ansammlung von mehr als einem Wert.
    Als Größen werden die einzelnen diskreten Werte $x_{i}$, sowie die Anzahl $N$ der Werte benötigt. \\
    Er wird wie folgt berechnet: $\overline{x}=\frac{1}{N}\sum_{i=1}^{N}x_i$. \\
    \subsection{Welche Bedeutung hat die Standardabweichung?} Die Standardabweichung bemisst die Messunsicherheit. 
    Wir bezeichnen die Standardabweichung als $\sigma$.
    68,3\% aller Ereignisse liegen bei einer Gauß-Verteilung in einem Bereich von einem $\sigma$. \\
    Je größer $\sigma$ ist, desto größer ist auch die Messunsicherheit, bzw. andersherum. 
    \subsection{Worin unterscheidet sich die Streuung der Messwerte und der Fehler des
    Mittelwertes?} Der Unterschied beläuft sich auf den Ursprung des jeweiligen Fehlers. Der Fehler des Mittelwerts 
    ist systematischen Ursprungs, d.h. er berücksichtigt Fehler im Versuchsaufbau. Er ist bei einer
    Messreihe in jedem Messwert enthalten. Die Streeung der Messwerte um den Mittelwert können z.B. händische Ungenauigkeiten,
    sich ändernde Bedingungen zwischen den Messwerten oder auch Ablesefehler sein. Dies ist ein Messfehler.
\end{document}