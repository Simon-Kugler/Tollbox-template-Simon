\section{Durchführung}
\label{sec:Durchführung}
Für den quadratischen sowie den runden Stab wird jeweils exakt das gleiche Messverfahren 
angewandt. Daher wird in der Durchführung nur einmal das Vorgehen erläutert. 
Es wird zuerst die Biegung des runden Stabes untersucht, darauf folgend dann die 
des quadratischen. Zu Anfang wird der Stab gewogen und in seiner Länge und seinem 
Durchmesser (der Breite) vermessen. Zum Messen der Länge wird ein Maßband verwendet,
für die Breite eine Schieblehre. Der Stab wird zuerst einseitig eingespannt. Es wird 
vermessen, wie viel des Stabes sich innerhalb der Messskala und außerhalb der Einspannung 
befindet. Dann wird eine Messuhr ans äußere Ende des Stabes geschoben. Ebenso wird eine
Hakenstange mit Gewichten bestückt. Jede Messreihe besteht aus dem Abstand zum hinteren 
Ende der Einspannung, der Auslenkung des Messstiftes ohne einghangenes Gewicht sowie Messung 
der Auslenkung mit eingehangenem Gewicht. An 16 Stellen werden zu $\qty{3}{\centi\meter}$-Abstäden 
die genannten Daten gemessen. Speziell bei der Messung des einseitig eingespannten runden 
Stabes musste gesondert auf die Position des Stabes bezüglich einer Drehung um die 
Längsachse geachtet werden. Näheres dazu in \autoref{sec:Diskussion}. Zur 
beidseitigen Einspannung des Stabes ändert sich an der Arretierung auf der rechten 
Seite nichts. Links wird der Stab mit einer kleinen Kerbe auf eine Kante gelegt. 
Von links übt eine Schraube mit einem flachen Mutteraufsatz Druck auf den Stab aus,
um ihn in Position zu halten. Die Stelle des Stabes, welche sich mittig beider 
Einspannungsenden befindet, wird markiert. Daraufhin werden in $\qty{2}{\centi\meter}$
-Schritten mit beiden Messuhren, jeweils links und rechts der Mitte, die gleichen 
Daten wie bei einseitiger Spannung abgelesen. Das Gewicht befindet sich in der Mitte 
des Stabes. Das gleiche Verfahren wird für den eckigen Stab wiederholt. 