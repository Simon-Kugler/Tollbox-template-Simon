\section{Diskussion}
\label{sec:Diskussion}
Zusammengefasst ergeben sich für die verschiedenen Messreihen die hier aufgeführten 
Elastizitätsmodule
\begin{align*}
    E_{\circ,\text{einseitig}}&=\qty{146(8,2)}{\giga\pascal}\,,\\
    E_{\circ,\text{beidseitig,rechts}}&=\qty{2.1(0.4)e2}{\giga\pascal}\,,\\
    E_{\circ,\text{beidseitig,links}}&=\qty{2.6(0.5)e2}{\giga\pascal}\,,\\
    E_{\square,\text{einseitig}}&=\qty{1,13(0.026)e2}{\giga\pascal}\,,\\
    E_{\square,\text{beidseitig, rechts}}&=\qty{4.6(1.1)e2}{\giga\pascal}\text{ und}\\
    E_{\square,\text{beidseitig, links}}&=\qty{4(4)e2}{\giga\pascal}\,.\\
\end{align*}
Die Flächenträgheitsmomente berechneten sich jeweils zu 
\begin{align*}
    I_\circ&=\qty{4,91(0.10)e-10}{\meter\tothe{4}}\text{ und}\\
    I_\square&=\qty{8,33(0,17)e-10}{\meter\tothe{4}}\,.
\end{align*}
Der Theoriewert für den Elastizitätsmodul von Kupfer liegt zwischen $100$ und $\qty{130}{\kilo\newton\per\milli\meter\squared}$\,
\cite{Elastizitätsmodul}.
Für die Berechnung der Abweichungen wird der mittlere Wert $\qty{120}{\kilo\newton\per\milli\meter\squared}$ genutzt.
Für die einseitige Einspannung des runden Stabes ergibt sich eine Abweichung von $21,6\%$ zum Theoriewert. Der Theoriewert liegt außerhalb 
des errechneten Fehlers. Wie bereits in der Durchführung erwähnt, stellte sich das passende Einspannen des runden Stabes als sehr schwierig
heraus, da dieser schon ohne Lastanhängung in sich sehr verbogen war. Schließlich wurde er so eingespannt, dass die Verbiegung hauptsächlich
lateral war und damit vertikal die beste Linearität erreicht werden konnte. Jedoch trafen die Messuhren dann logischerweise nicht genau die 
Mitte des Stabes, was zu Fehlern geführt hat.
Die Abweichungen des beidseitig eingespannten runden Stabes liegen bei $175\%$ sowie $217\%$. Hier liegen die Messdaten ebenfalls außerhalb des 
Fehlerintervalls.
Der quadatische Stab war deutlich weniger vorgebogen als der runde. Dies spiegelt sich auch in der Messung der einseitigen Einspannung wider.
Die Abweichung liegt hier bei $5,8\%$\,. Damit ist diese Messreihe die genaueste. Ebenfalls ist das der einzige Wert, der im Intervall
des Theoriwerts liegt. Die ungenauesten Ergebenisse liefert die Messreihe der beidseitigen Einspannung des qaudratischen Stabes.
Insbesondere für die linke Seite ist der Fehler der Berechnung bzw. des Koeffizienten der linearen Regression so groß, dass kaum eine Aussage über
die Gültigkeit dieses Ergebnisses getroffen werden kann. Die Abweichungen liegen bei $380\%$ sowie $330\%$. Durch die Größe der Fehler
sind diese jedoch eher Werte zur Orientierung als zur genaueren Analyse nützlich.  
Als Fehlerquellen sind vor allem jene systematischer Art hervorzuheben. Die Messuhren sind sehr sensibel für Vibrationen etc. Die Messung 
wurde also zum Beispiel durch eine am Versuch vorbeilaufenden Person verfälscht. Andererseits hingen die Stifte der Uhren in manchen Positionen leicht 
fest und konnten nur durch Erschütterungen wieder gelöst werden. Als Erklärung für die großen Fehler bzw Abweichungen der beidseitigen Einspannung 
könnten die kleineren Auslenkungen als bei einseitiger Einspannung möglich sein. Ebenfalls bereitet die Durchbiegung des Stabes ohne Last in Richtungen, 
die nicht in die Messreihe mit aufgenommen wurden einen gewissen Fehler. Statistische Fehler könnten mit mehr Messungen minimiert werden. 
Bei beidseitiger Durchbiegung wurden jeweils nur 10, bei einseitiger 16 Messerte aufgenommen.
