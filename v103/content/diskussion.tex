\section{Diskussion}
\label{sec:Diskussion}
Zusammengefasst ergeben sich für die verschiedenen Messreihen die hier aufgeführten 
Elastizitätsmodule
\begin{align*}
    E_{\circ,\text{einseitig}}&=\qty{146(8,2)}{\giga\pascal}\,,\\
    E_{\circ,\text{beidseitig,rechts}}&=\qty{2.1(0.4)e2}{\giga\pascal}\,,\\
    E_{\circ,\text{beidseitig,links}}&=\qty{2.6(0.5)e2}{\giga\pascal}\,,\\
    E_{\square,\text{einseitig}}&=\qty{1,13(0.026)e2}{\giga\pascal}\,,
    E_{\square,\text{beidseitig, rechts}}&=\qty{4.6(1.1)e2}{\giga\pascal}\,,\\
    E_{\square,\text{beidseitig, links}}&=\qty{4(4)e2}{\giga\pascal}\,.\\
\end{align*}
Die Flächenträgheitsmomente berechneten sich jeweils zu 
\begin{align*}
    I_\circ&=\qty{4,91(0.10)e-10}{\meter\tothe{4}}\,,\\
    I_\square=\qty{8,33(0,17)e-10}{\meter\tothe{4}}\,.
\end{align*}
Der Theoriewert für den Elastizitätsmodul von Kupfer liegt zwischen $100$ und $\qty{130}{\kilo\newton\per\milli\meter\squared}$\,.
Für die Berechnung der Abweichungen wird der mittlere Wert $\qty{120}{\kilo\newton\per\milli\meter\squared}$ genutzt.
Für die einseitige E