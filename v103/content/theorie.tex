\section{Zielsetzung}
Das Ziel dieses Versuches ist es das Elastizitätsmodul verschiedener Metalle zu bestimmen.

\section{Theorie}
\label{sec:Theorie}

Wenn Kräfte an Oberflächen von Körpern angreifen, können diese Gestalts- und Volumenveränderungen hervorrufen.
Diese Kräfte bezogen auf die Flächeneinheit heißt Spannung.
Die senkrecht zur Oberflächen verlaufende Komponente heißt Normalspannung $\sigma$ und die parallele Komponente wird Tangentialspannung genannt.
Die Formänderung kann durch die relative Änderung $\sfrac{\delta L }{L}$, mit einer linearen Körperdimension $L$, beschreiben werden.
Ist diese Änderung hinreichend klein, entsteht eine lineare Beziehung zwischen $\sfrac{\delta L }{L}$ und der Spannung $\sigma$.
Diese Beziehung heißt das Hookesche Gesezt und hat die Form
\begin{equation}
    \sigma = E \frac{\delta L}{L}
\end{equation}
Hierbei ist $E$ das Elastizitätsmodul, welches eine Materialkonstante ist.
Um diese zu bestimmen, wird eine spezielle Deformation verwendet, die sogenannte Biegung.
\subsection{Durchbiegung eines homogenen Stabes bei einseitiger Einspannung}
%Hier ABB 2 aus der Theorie einsetzen
Der Versuchsaufbau ist in %\autoref{}
dargestellt.
Es soll die Durchbiegung $D(x)$, also die Verschiebung eines Oberflächenpunktes $x$ zwischen belastet und unbelastet errechent werden.
Mithilfe von $D$ lässt sich mit den gemessenen Größen $E$ bestimmen.
Wie in %\autoref
zu sehen, wird der Querschnitt $Q$ aus seiner vertikalen Lage durch die Kraft $F$ verdreht.
Auf ihn wirkt das Drehmoment $M_\text{F}$.
Auf die obere Schicht des Stabes wirkt die Zugspannung, somit wird sie gedehnt.
Auf die untere Schicht dagegen wirkt die Drucksapnnung und sie wird gestaucht.
Die Schicht dazwischen wird weder gedehnt, noch gestaucht.
Diese wird als neutrale Faser bezeichnet und ist in %\autoref{}
als gestrichelte Linie dargestellt.
%Hier Abb 3
In %\autoref{}
ist die Richtung der Zug- und Druckspannung in dem Stab dargestellt.
Dardurch, dass sie in entgegengetzte Richtungen zeigen, bewirken sie ein Drehmoment $M_\text{\sigma}$.
Durch ein einstellendes Gleichgewicht, können die beiden Drehmomente $M_\text{F}$ und $M_\text{\sigma}$ gleichgesetzt werden.
