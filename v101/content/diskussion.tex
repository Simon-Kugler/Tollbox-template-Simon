\section{Diskussion}
\label{sec:Diskussion}
Im Versuch bzw. der Auswertung wurden insgesamt 5 Trägheitsmomente und eine Winkelrichtgröße bestimmt.
4 dieser Werte lassen sich davon mit theoretisch errechneten Werten abgleichen. Da das Eigenträgheitsmoment
$I_D$ der Drillachse zu vernachläsigen war, sind alle experimentell bestimmten Trägheitsmomente deutlich größer
als die theoretisch berechneten. Für die Kugel ergibt sich für das Verhältnis 
$\sfrac{I_\text{exp}}{I_\text{theo}}=0.69$, also eine Abweichung von $31\%$ des experimentellen zum theoretischen Wert.
Analog dazu ergibt sich für den Zylinder eine Abweichung von $32\%$. Die beiden Fehler liegen also sehr nah beieinander.
Für die zwei Haltungen der Puppe ergeben sich ebenso erneut experimentell deutlich kleinere Werte als theoretisch berechnet.
Für die erste Haltung lässt sich eine Abweichung von $65\%$ feststellen, für die Haltung mit ausgestreckten Beinen
$42\%$. Damit sind die Fehler hier deutlich höher, was mehrere Ursachen haben könnte. Erstens werden für die Puppe einige 
Näherungen angenommen. Zunächst passt die Annäherung eines an einen Zylinder nicht unbedingt gut, ebenso werden die Mettalstäbe,
Gelenke, Füße und der Hals vernachlässigt. Der Winkel der abgeknickten Beine in Haltung 2 wird nicht genau 90° gewesen sein, 
was den Abstand der Masse zum Körper verringert. Ebenso schwingt die Puppe deutlich schneller als die Kugel oder der
Zylinder. Dies erschwert ein genaues Starten und Stoppen der Zeitmessung.\\ \noindent
Eine Fehlerquelle für alle Messreihen ist die nicht reibungsfreie Apparatur. Diese verkürzt die Schwingungsdauern deutlich gegenüber
der reibungsfreien theoretischen Berechnung. Dadurch verkleinert sich das Trägheitsmoment. Ebenso war die Messung für die 
Winkelrichtgroesse sehr schwierig, da gleichzeitig auf ein senkrechtes Halten des Kraftmessers an der Stange als auch 
gleichzeitig auf den richtig eingestellten Winkel geachtet werden musste. Ebenso wurde in der Messreihe der Kraftmesser gewechselt, 
da jeder nur einen gewissen Messbereich abdecken kann. Da $D$ für alle weiteren Berechnungen benötigt wurde, kann dies großen Einfluss
auf die Qualität der Werte bewirken.\\ \noindent
Das Eigenträgheitmoments der Drillachse wird aufgrund der schon aufgeführten Gründe ebenso fehlerbehaftet sein. Dies lässt sich jedoch
aufgrund einer fehlenden theoretischen quantitativen Größe nicht einordnen.\\ \noindent
Abschließend lassen sich bei allen Messungen große Unsicherheiten feststellen, jedoch sind diese immer in einer Größenorndung mit dem 
Theoriewert, was auf viele systematische Fehler und Ablesefehler schließen lässt, der Versuch jedoch richtig durchgeführt wurde.       