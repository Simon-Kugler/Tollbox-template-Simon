\section{Theorie}
\label{sec:Theorie}
Ziel des Versuches ist es, das Trägheitsmoment verschgiedener Körper
experimentell zu bestimmen, teilweise unter Benutzung des Satz von Steiner.
\\ \noindent
Die für den Versuch wichtigste Größe ist das Trägheitsmoment, welche sich durch die Masse $m$
eines Körpers sowie seinen den Abstand $r$ seiner Ausdehnung zur Drehachse zusammensetzt. Je schwerer die Massen sind und je größer der Abstand 
zur Drehachse ist, desto größer ist das Trägheitsmoment. Dies wird für eine diskrete Massenverteilung durch  
\begin{equation}
    I=\sum_i r_i^2\cdot m_i
\end{equation}
beschrieben.
Für kontinuierliche Verteilungen lässt sich $I$ durch 
\begin{equation}
    I=\int r^2\, dm
\end{equation}
ausdrücken. Für nur eine Masse mit homogener Masservetilung und einem konstanten mittleren Abstand $r$ zur Drehachse gilt damit 
$I=mr^2$. \\ \noindent
Liegt ein Problem vor, bei dem die Drehachse, für welche das Trägheitsmoment bestimmt werden soll,
nicht durch den Schwerpunkt des Körpers verläuft, kann der Satz von Steiner genutzt werden: 
\begin{equation}
    I=I_S+m\cdot a^2.
\end{equation}
$I$ ist das Trägheitsmoment um die verschobene, aber zur ursprünglichen Achse parallelen Achse,
$m$ die Masse des Körpers. $I_S$ ist das schon bestimmte Trägheitsmoment um die Drehachse und $a$
beschreibt die Länge zwischen beiden Achsen.\\ \noindent
Das Drehmoment $\vec M$ setzt sich aus einer Kraft $\vec F$ die im Abstand $\vec r$ am Rotationskörper 
angreift zusammen:
\begin{equation}
    \vec M=\vec F\times\vec r.
\end{equation}
Wird ein sogenanntes \glqq schwingungsfähiges\grqq\, System um einen Winkel $\varphi$ ausgelenkt,
stellt sich eine Schwingung mit der  Periodendauer 
\begin{equation}
    T=2\pi\sqrt{\frac{I}{D}}.
\end{equation}
$D$ ist hierbei die Winkelrichtgröße. Damit lässt sich ebenso das Drehmoment $M$ berechnen 
\begin{equation}
    M=D\cdot \varphi.
\end{equation}
Diese Beschreibung gilt nur für kleine Auslenkungrn $\varphi$.