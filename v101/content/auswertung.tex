\section{Auswertung}
\label{sec:Auswertung}

Zunächst wird aus den Messwerten die Winkelrichtgröße $D$ bestimmt.
Hierfür werden die Drehwinkel $\phi$ zu nächst von $^\circ$ in $rad$ umgerechnet.
Man erhält die Formel $\phi_{neu} = \phi \sfrac{\pi}{180}$.
$D$ wird dann mit
\begin{equation}
  D = \frac{F r}{\phi}
\end{equation}
bestimmt.
Dabei ist $F$ die gemessene Kraft und $r$ der Radius, an dem der Kraftmesser befestigt wird.
In diesem Versuch wird $r = \Si{0.2}{\meter}$ gewählt.
Es werden die Winkelrichtgrößen aller Winkel einzeln ausgerechnet.
Anschließend wird der Mittelwert dieser gebildet und $D = \Si*{0.01997}{\N \meter \per\rad}$ ergibt sich.
