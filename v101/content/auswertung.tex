\section{Auswertung}
\label{sec:Auswertung}

Zunächst wird aus den Messwerten die Winkelrichtgröße $D$ bestimmt.
\begin{table}
  \centering
  \caption{Messwerte zur Bestimmung der Winkelrichtgröße}
  \label{tab:Winkelrichtgroesse}
  \begin{tabular}{
  c c
  }
    \toprule
     $\phi \, / \unit{°}$ & $F\, / \unit{\newton}$\\
    \midrule
    110 & 0.21 \\
    100 & 0.19 \\
    90  & 0.188\\
    80  & 0.168\\
    70  & 0.147\\
    60  & 0.118\\
    50  & 0.090\\
    40  & 0.060\\
    30  & 0.038\\
    20  & 0.016\\
    \bottomrule
  \end{tabular}
\end{table}
Hierfür werden die Drehwinkel $\phi$ zu nächst von $^\circ$ in $rad$ umgerechnet, was mit $\phi_\text{neu} = \sfrac{\phi \pi}{180}$ durchgeführt wird.
$D$ wird dann mit
\begin{equation}
  D = \frac{F R}{\phi_\text{neu}}
\end{equation}
bestimmt.
Dabei ist $F$ die gemessene Kraft und $R$ der Radius, an dem der Kraftmesser befestigt wird.
In diesem Versuch wird $R = \SI{0.2}{\meter}$ gewählt.
Es werden die Winkelrichtgrößen aller Winkel einzeln ausgerechnet.
Anschließend wird der Mittelwert dieser gebildet und $D = \SI{0.01997}{\N \meter}$ ergibt sich.

\begin{table}
  \centering
  \caption{Messwerte zur Bestimmung des Trägheitsmoments}
  \label{tab:Traegheitsmoment}
  \begin{tabular}{
  c c
  }
    \toprule
     $R \, / \unit{\meter}$ & $F\, / \unit{\newton}$\\
    \midrule
    0.05  & 14.16 \\
    0.075 & 15.97 \\
    0.10  & 18.34 \\
    0.125 & 20.78 \\
    0.15  & 23.47 \\
    0.175 & 26.47 \\
    0.20  & 29.81 \\
    0.225 & 32.81 \\
    0.25  & 35.82 \\
    0.30  & 41.79 \\
    \bottomrule
  \end{tabular}
\end{table}


\begin{table}
  \centering
  \caption{5-fache Schwingungsdauer einer Kugel und einers Zylinders}
  \label{tab:Kugel_Zylinder}
  \begin{tabular}{
  c c
  }
    \toprule
     $T_\text{Kugel}\, / \unit{\second}$ & $T_\text{Zylinder}\, / \unit{\second}$\\
    \midrule
    9.15 & 9.47 \\
    9.31 & 9.43 \\
    9.19 & 9.31 \\
    9.34 & 9.47 \\
    9.28 & 9.31 \\
    9.35 & 9.28 \\
    9.41 & 9.41 \\
    9.28 & 9.31 \\
    9.31 & 9.28 \\
    9.35 & 9.28 \\
    \bottomrule
  \end{tabular}
\end{table}