\section{Durchführung}
\label{sec:Durchführung}
Die Apparatur besteht aus einem Fuß mit drehbarer Scheibe, welche durch eine Feder in einer bestimmten 
Position gehalten wird. Wird die Scheibe um einen Winkel $\varphi$ gedreht, so wirkt proportional zur Auslenkung 
eine rückstellende Kraft. Mit dem Drehmechanismus ist eine Aufnahme für Stangen fest verbunden, um entweder
senkrecht oder waagerecht eine Stange festzustecken.  
Zu Anfang des Versuches wurde die Winkelrichtgröße $D$ dieser Apparatur bestimmt. Hierzu wird waagerecht eine Stange 
eingesteckt, dessen Mittelpunkt in der Mitte der Scheibe liegt. Dann wird die Stange um 10 verschiedene Winkel bis zu 
$120°$ ausgelenkt und mit einem Federkraftmesser die dafür benötigte Kraft benötigt. Der Federkraftmesser wird 
in einem Abstand von $20$\,cm zum Mittelpunkt angesetzt. Die nächste zu bestimmende Messgröße ist das 
Eigenträgheitsmoment der Drillachse $I_D$. Dazu werden zwei Gewichte, dessen Gewicht und Maße vorher gemessen werden,
auf der Stange so befestigt, dass sie jeweils den gleichen Abstand zum Mittelpunkt haben.
Für 10 verschiedene Abstände zum Mittelpunkt der Gewichte wird für eine Startauslenkung 
von $90°$ die Schwingungsdauer $T$ für 5 Schwingungen gemessen. Es wird im ganzen Versuch die 5-fache Dauer gemessen, 
da dies den Fehler bei sehr kleinen Schwingungsdauern minimiert.
\subsection{Trägheitsmoment einer Kugel}
Zum Bestimmen des Trägheitsmoment einer Kugel wird diese zunächst vermessen und gewogen. Dann wird sie über einen mittig
angebrachten Stab auf die Apparatur gesteckt. Sie wird 10 mal um einen Winkel von 90° ausgelenkt um dann jeweils wieder 
die 5-fache Schwingungsdauer zu messen. 
\subsection{Trägheitsmoment eines Zylinders}
Um das Trägheitsmoment eines Zylinders zu bestimmen, wird analog zum Verfahren der Bestimmung des Trägheitsmoment einer Kugel vorgegangen.
\subsection{Trägheitsmoment einer Holzpuppe}
Zunächst wird auch die Puppe wieder vermessen. Es werden Maße der Arme, Beine, des Kopfes und des Torsos gemessen. Die 
Höhe bzw. Länge wird nur einmal gemessen, der Durchmesser der Arme, Beine und des Torsos zehn mal. Der Durchmesser des 
Kopfes wird 5 mal gemessen. Zum Messen aller Größen wird eine Schieblehre verwendet. Das Tägheitsmoment der Puppe wird 
mit zwei unterschiedlichen Haltungen der Puppe bestimmt. Sie ist so befestigt, als würde sie aufrecht auf der Drillachse 
stehen. In der ersten Position sind die Arme zur Seite ausgestreckt. In der zweiten Haltung sind dann zusätzlich die 
Beine nach vorn gestreckt. Die Schwingungsdauer wird insgesamt 20 mal gemessen. Jeweils 10 pro Haltung sowie jeweils 
5 mal für die Auslenkung von $90$ und $120°$.