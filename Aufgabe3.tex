\documentclass[titlepage=firstiscover, parskip=half, bibliography=totoc,captions=tableheading]{scrartcl} % KOMA-Script Dokumentenklasse Article

% Warnung, falls noch einmal kompiliert werden muss
\usepackage[aux]{rerunfilecheck}

% Paket für Schriftarteinstellung, muss immer geladen werden
\usepackage{fontspec}

% Deutsche Spracheinstellungen, wichtig z. B. für korrekte Trennung
\usepackage[ngerman]{babel}

% mehr Pakete hier
\usepackage{amsmath} % unverzichtbare Mathe-Befehle
\usepackage{amssymb} % viele Mathe-Symbole
\usepackage{mathtools} % Erweiterungen für amsmath
\usepackage[
math-style=ISO, % ┐
bold-style=ISO, % │
sans-style=italic, % │ ISO-Standard folgen
nabla=upright, % │
partial=upright, % │
mathrm=sym, % ┘
]{unicode-math}
\usepackage[
version=4,
math-greek=default,
text-greek=default,
]{mhchem}
\usepackage[section, below]{placeins}
\usepackage{caption} % Captions schöner machen
\usepackage{graphicx}
\usepackage{graphicx}
\usepackage{tabularray}
\usepackage[
locale=DE,
separate-uncertainty=true, % Immer Unsicherheit mit ±
per-mode=symbol-or-fraction, % m/s im Text, sonst \frac
% alternativ:
% per-mode=reciprocal, % m s^{-1}
% output-decimal-marker=., % . statt , für Dezimalzahlen
]{siunitx}

\UseTblrLibrary{booktabs}
\UseTblrLibrary{siunitx} % Lädt siunitx und definiert die S-Spalte
\usepackage[style=alphabetic]{biblatex} % nach babel
\addbibresource{lit.bib}

% Unterstützung für Links und PDF Metadaten
\usepackage[unicode]{hyperref}
\usepackage{bookmark}
\usepackage{microtype}
\usepackage{xfrac}
% Einstellungen hier, z.B. Fonts
\setmathfont{Latin Modern Math}
% \setmathfont{Tex Gyre Pagella Math} % alternativ




\begin{document}
\title{Abgabe 3}
\author{
Sophia Brechmann \\
\texorpdfstring{\href{mailto:sophia.brechmann@tu-dortmund.de}{sophia.brechmann@tu-dortmund.de}\and}{,}
Simon Kugler \\
\texorpdfstring{\href{mailto:simon.kugler@tu-dortmund.de}{simon.kugler@tu-dortmund.de}}{}
}
\date{Deadline: Dienstag, 07.11.2023}
\maketitle
\section{Volumen eines Hohlzylinders: }
\begin{align*}
    R_\text{innen}&=\qty{10+-1}{\centi\meter}\,\,R_\text{außen}=\num{10+-1}\unit{\centi\meter},\,h=(20\pm1)\text{cm}\\
    V&=(\pi R_\text{außen}^2-\pi R_\text{innen}^2)\cdot h 
    \intertext{Der wahrscheinlichste Fehler, für eine von mehreren Variablen 
    abhängigen Funktion, berechnet sich wie folgt: }
    % \Delta f_\text{max}&=\big\vert\frac{df}{dy_1}\big\vert\Delta y_1+...+
    % \big\vert\frac{df}{dy_N}\big\vert\Delta y_N \\
    % \Delta f_\text{max}&=\big\vert\frac{df}{dR_\text{außen}}\big\vert\Delta R_\text{außen}+
    % \big\vert\frac{df}{dR_\text{innen}}\big\vert\Delta R_\text{innen}+
    % \big\vert\frac{df}{dh}\big\vert\Delta h \\
    % &=\big\vert 2\pi R_\text{außen}\cdot h\big\vert\cdot 1+\big\vert 2\pi R_\text{innen}\cdot h\big\vert\cdot 1+
    % \big\vert \pi R_\text{außen}^2-\pi R_\text{innen}^2\big\vert \cdot 1\\
    % &=2\pi\cdot 15\cdot20+2\pi\cdot 10\cdot 20+\pi\cdot 15^2-\pi\cdot 10^2\\
    % \Delta f_\text{max}&=1125\pi...???\\
    \Delta f&=\sqrt{\sum_{i=1}^N\left(\frac{df}{dy_i}\right)^2(\Delta y_i)^2}\\
    \intertext{Angewandt bzw. eingesetzt in unsere Formel: }
    \Delta f&=\huge{[}\sqrt{(2\pi\cdot15\unit{cm}\cdot 20\unit{cm})^2\cdot (1\unit{cm})^2+(-2\pi\cdot 10\unit{cm}\cdot 20\unit{cm})^2\cdot (1\unit{cm})^2}\\
    &\overline{(\pi\cdot(15\unit{cm})^2-\pi\cdot(10\unit{cm})^2)^2\cdot (1\unit{cm})^2}\huge]\\
    &\approx2299,22\unit{cm^3}\\
    V&=(\pi\cdot(15\unit{cm})^2-\pi(10\unit{cm})^2)\cdot20\unit{cm}\\
    &\boxed{=(7583,98\pm2299,22)\unit{cm^3}}
\end{align*}
\section{Kinetische Energie eines Projektils}
Die kinetische Energie $E_\text{kin}$ berechnet sich wie folgt:
\begin{align*}
    E_\text{kin}&=\frac{1}{2}mv^2\\
    \intertext{Mit obiger Formel ergibt sich für $\Delta f$: }
    \Delta f&=\sqrt{(\frac{1}{4}\cdot (200\unit{m/s})^4)\cdot (0.0001\unit{kg})^2+((0,005\unit{kg})^2\cdot(200\unit{m/s})^2)\cdot (10\unit{m/s})^2}\\
    &=10\unit{J}\\
    \intertext{Damit ist die kinetische Energie: }
    E_\text{kin}&=\frac{1}{2}mv^2=\frac{1}{2}\cdot 0,005\unit{kg}\cdot200\unit{m\s}^2\\
    &\boxed{=(100\pm 10)\unit{J}}\\
    \intertext{Die zurückgelegte Strecke berechnet sich über: }
    s&=v\cdot t \\
    \Delta f&=\sqrt{((6\unit{s})^2)\cdot (10\unit{m/s})^2+((200\unit{m/s})^2)\cdot(0\unit{s})^2}\\
    &=60\unit{m}\\
    s&=v\cdot t=200\unit{m/s}\cdot 6\unit{s}\\
    &\boxed{=(1200\pm 60)\unit{m}}
\end{align*}
\end{document}