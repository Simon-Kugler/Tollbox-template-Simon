\documentclass[titlepage=firstiscover, parskip=half, bibliography=totoc,captions=tableheading]{scrartcl} % KOMA-Script Dokumentenklasse Article

% Warnung, falls noch einmal kompiliert werden muss
\usepackage[aux]{rerunfilecheck}

% Paket für Schriftarteinstellung, muss immer geladen werden
\usepackage{fontspec}

% Deutsche Spracheinstellungen, wichtig z. B. für korrekte Trennung
\usepackage[ngerman]{babel}

% mehr Pakete hier
\usepackage{amsmath} % unverzichtbare Mathe-Befehle
\usepackage{amssymb} % viele Mathe-Symbole
\usepackage{mathtools} % Erweiterungen für amsmath
\usepackage[
math-style=ISO, % ┐
bold-style=ISO, % │
sans-style=italic, % │ ISO-Standard folgen
nabla=upright, % │
partial=upright, % │
mathrm=sym, % ┘
]{unicode-math}
\usepackage[
version=4,
math-greek=default,
text-greek=default,
]{mhchem}
\usepackage[section, below]{placeins}
\usepackage{caption} % Captions schöner machen
\usepackage{graphicx}
\usepackage{graphicx}
\usepackage{tabularray}
\usepackage[
locale=DE,
separate-uncertainty=true, % Immer Unsicherheit mit ±
per-mode=symbol-or-fraction, % m/s im Text, sonst \frac
% alternativ:
% per-mode=reciprocal, % m s^{-1}
% output-decimal-marker=., % . statt , für Dezimalzahlen
]{siunitx}

\UseTblrLibrary{booktabs}
\UseTblrLibrary{siunitx} % Lädt siunitx und definiert die S-Spalte
\usepackage[style=alphabetic]{biblatex} % nach babel
\addbibresource{lit.bib}

% Unterstützung für Links und PDF Metadaten
\usepackage[unicode]{hyperref}
\usepackage{bookmark}
\usepackage{microtype}
\usepackage{xfrac}
% Einstellungen hier, z.B. Fonts
\setmathfont{Latin Modern Math}
% \setmathfont{Tex Gyre Pagella Math} % alternativ




\begin{document}
\begin{table}
    \centering
    \caption{Temperatur und Druck bei Verdampfung des Wassers. Der Druck hat eine Messunsicherheit von
    $\pm$mB, die Temperatur von $\pm 1$ K.},
    \label{tab:Messreihe_1}
\begin{tabular}{
    c c||c c||c c||c c
}
\toprule 
$T/ \unit{\kelvin}$ & $p / \text{mB}$ & $T/ \unit{\kelvin}$ & $p / \text{mB}$&
$T/ \unit{\kelvin}$ & $p / \text{mB}$ & $T/ \unit{\kelvin}$ & $p / \text{mB}$ \\
\midrule
293,15  & 40   & 313,15  & 372  & 333,15  & 528  & 353,15  & 773 \\
294,15  & 167  & 314,15  & 379  & 334,15  & 537  & 354,15  & 790 \\
295,15  & 228  & 315,15  & 385  & 335,15  & 549  & 355,15  & 810 \\
296,15  & 255  & 316,15  & 395  & 336,15  & 562  & 356,15  & 826 \\
297,15  & 274  & 317,15  & 400  & 337,15  & 571  & 357,15  & 844 \\
298,15  & 287  & 318,15  & 405  & 338,15  & 578  & 358,15  & 856 \\
299,15  & 296  & 319,15  & 412  & 339,15  & 586  & 359,15  & 879 \\
300,15  & 303  & 320,15  & 421  & 340,15  & 600  & 360,15  & 901 \\
301,15  & 310  & 321,15  & 431  & 341,15  & \text{--}  & 361,15  & 913 \\
302,15  & 316  & 322,15  & 439  & 342,15  & 628  & 362,15  & 933 \\
303,15  & 322  & 323,15  & 445  & 343,15  & 638  & 363,15  & 944 \\
304,15  & 327  & 324,15  & 453  & 344,15  & 650  & 364,15  & 966 \\
305,15  & 331  & 325,15  & 461  & 345,15  & 660  & 365,15  & 979 \\
306,15  & 336  & 326,15  & 465  & 346,15  & 670  & 366,15  & 990 \\
307,15  & 341  & 327,15  & 476  & 347,15  & 690  & 367,15  & 1015\\
308,15  & 347  & 328,15  & 482  & 348,15  & 701  & 368,15  & 1030\\
309,15  & 351  & 329,15  & 491  & 349,15  & 716  & 369,15  & 1046\\
310,15  & 356  & 330,15  & 500  & 350,15  & 723  & 370,15  & 1061\\
311,15  & 362  & 331,15  & 511  & 351,15  & 743  & 371,15  & 1079\\
312,15  & 367  & 332,15  & 520  & 352,15  & 760  & 372,15  & 1093\\
        &   &       &       &       &      & 373,15  & 1108 \\
\bottomrule
\end{tabular}
\end{table}
\begin{table}
    \centering
    \caption{Temperatur und Druck bei Verdampfung des Wassers.},
    \label{tab:Messreihe_1}
\begin{tblr}{
    colspec = {S S}
}
\toprule 
$p$/B & $T$/K\\
\midrule
1000  & 391,15\\
2000  & 368,15\\
3000  & 413,15\\
4000  & 419,15\\
5000  & 427,15\\
6000  & 433,15\\
7000  & 438,15\\
8000  & 443,15\\
9000  & 447,15\\
10000 & 451,15\\
11000 & 454,15\\
12000 & 459,15\\
13000 & 188,15\\
14000 & 461,15\\
15000 & 465,15\\
\bottomrule
\end{tblr}
\end{table}



\end{document}