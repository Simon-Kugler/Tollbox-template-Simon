    \section{Zielsetzung}
    Ziel dieses Versuchs ist die Bestimmung einzelner Bauteile im elektrischen Schaltkreis,
    sowie die Ermittlung der Frequenzabhängigkeit der Brückenspannung einer Wien-Robinson-Brücke.

\section{Theorie}
\label{sec:Theorie}
Die Inhalte des Theorieteils sind auf Grundlage des Dokuments \cite{V302_Anleitung} zusammengetragen.
Brückenschaltungen sind eine essentielle Messmethode, da sie sehr viel präziser sind als herkömmliche Methoden.
Besondere Bedeutung bekommt dabei die Nullmethode, welche die zu messende Größe mit einer hohen Genauigkeit bestimmt,
indem sie die Spannung abgleicht.
Dabei ist es wichtig, dass sich jede physikalische Größe durch Widerstände ausdrücken lässt.
\subsection{Allgemeine Brückenschaltung}
Bei der allgemienen Brückenschaltung sind alle Widerstände bekannt und die Brückenspannung zwischen A und B (siehe
Abbildung \ref{})kann berechnet werden.
Dafür werden die Kirchhoffschen Gesetzte verwendet.
Diese lauten:
\begin{enumerate}
    \item Die Kontenregel besagt, dass die Summe der zufließenden Ströme gleich der Summe der abfließenden Ströme ist.
    \item Die Maschenregel besagt, dass in jedem in sich geschlossenen Stromkreis die Summe der Spannung gleich null ist.
\end{enumerate}



