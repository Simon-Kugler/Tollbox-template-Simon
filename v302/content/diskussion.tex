\newpage
\section{Diskussion}
\label{sec:Diskussion}
\noindent
Die Fehler der Messwerte der Messreihen mit konstanter Frequenz sind sehr unterschiedlich.
Bei einigen ist der Fehler sehr klein, bei anderen sehr groß. Die kleinste Abweichung 
lässt sich mit $5,2\%$ für den zweiten Widerstand der Kapatitätsmessbruecke feststellen.
Die größte Abweichung beträgt $100\%$ für den Widerstand der Maxwell-Brücke. Der wohl größte 
systematische Fehler ist das Ablesen der Nullinie auf dem Oszilloskop, welche sich je nach Skala
nicht genau bestimmen ließ. Dies hat sich dadurch bemerkbar gemacht, dass bei feinster Auflösung des
Oszilloskops eine augescheinliche Nullinie für ein Band von verschiedenen Widerständen zu erkennen war.
Als statistischer Fehler lässt sich das nicht variieren des Widerstandes $R_2$ einstufen. 
Durch wechseln dieses hätte sich der Fehler minimieren lassen können, jedoch wurde dies
aufgrund fehlender varibaler Widerstände nicht durchgeführt. Einer der beiden vorhandenen variablen 
Widerstände hat eine fehlerhafte Einstell- bzw Messskala, was das korrekte Justieren deutlich erschwert hat.
Der Ring, auf welchem sich der eingestellte Widerstand ablesen lässt und welcher im Normalfall fest mit dem
Drehknopf verbunden ist, war in diesem Fall nicht fest verbunden. Daher musste vorsichtig gedreht 
beziehungsweise abgelesen werden um ein ungewolltes Verrutschen des Rings zu vermeiden. 
Ein weiterer systematischer Fehler, dessen Einfluss jedoch schwer abzuschätzen ist, ist die Verbindung 
der Kabel miteinander. Je nach Schaltbild wurden drei Kabel an- bzw. ineinander gesteckt. Die Stecker der 
Kabel waren jedoch von unterschiedlichem Typ, was zum Beispiel ungewollte Widerstände hervorgebracht 
haben könnte. Dies wäre auch eine Erklärung für die Differenz zwischen den einzelnen Abweichungen.\\ \noindent
Der zweite Teil des Versuchs weist weniger fehlerbehaftete Größen auf. Ein Ablesefehler ist hier nicht vorhanden, da 
sowohl die zu justierende Frequenz, als auch die Amplitude am Oszilloskop digital als Zahl angezeigt wurden.
Die in Abbildung \ref{fig:plot} zu erkennende Theoriekurve bildet im Groben gut den Verlauf der Messwerte ab.
Im ersten Teil bis zum Wendepunkt der Kurve liegen die Messwerte sehr nah an der Theoriekurve, ab dort zeichnen
Messwerte und Theoriekurve einen leicht anderen Verlauf ab. Die Messwerte erreichen deutlich schneller ihren 
Grenzwert, jedoch liegt der grenzwert der Theoriekurve etwas höher. Dies lässt sich durch etwaige Störfrequenzen 
oder unbekannte Widerstände im System erklären. \\ \noindent
Der bereits berechnete Klirrfaktor $k$ liegt leicht über der gemessenen kleinsten Spannung, was genau so zu 
ewartetn war.