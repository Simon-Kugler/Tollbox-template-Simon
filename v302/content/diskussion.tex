\section{Diskussion}
\label{sec:Diskussion}
Die Fehler der Messwerte der Messreihen mit konstanter Frequenz sind sehr unterschiedlich.
Bei einigen ist der Fehler sehr klein, bei anderen sehr groß. Die kleinste Abweichung 
lässt sich mit $5,2\%$ für den zweiten Widerstand der Kapatitaetsmessbruecke feststellen,
die größte Abweichung betägt $100\%$ für den Widerstand der Maxwell-Brücke. Der wohl größte 
systematische Fehler ist das Ablesen der Nullinie auf dem Oszilloskop, welche sich, je nach Skala, 
nicht genau bestimmen ließ. Als statistischer Fehler lässt sich das nicht variieren des Widerstandes 
$R_2$ einstufen. Durch wechseln dieses hätte sich der Fehler minimieren lassen können, jedoch wurde dies
aufgrund fehlender varibaler Widerstände nicht durchgeführt. Einer der beiden vorhandenen variablen Widerstände
hat eine fehlerhafte Einstell- bzw Messskala, was das korrekte justieren deutlich erschwert hat.                                                                                             