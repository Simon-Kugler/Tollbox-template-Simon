\section{Durchführung}
\label{sec:Durchführung}
Der Versuchsaufbau ist gegeben durch ein digitales Oszilloskop, einen Funktionengenerator, verschiedenartige 
Widerstände sowie Kabel. Ebenso ist auch ein Tiefpass-Filter vorhanden, welcher aber nicht zwingend zur
Durchführung benötigt wird. Hauptaufgabe ist es, die Kondensatoren, Spulen sowie Ohmschen Widerstände
nach verschiedenen Schaltbildern aus Kapitel \ref{sec:Theorie} zusammenzuschließen. Damit werden dann jeweils die
Widerstände, Kapazitäten bzw. Induktivitäten bestimmt. Ist eine Schaltung aufgebaut, wird 
der verstellbare Widerstand so justiert, dass die auf dem Oszilloskop dargestellte Schwingung ihr Amplituden-
Minimum erreicht. An diesem Punkt, an dem die Amplitude genähert gleich null ist, gelten 
die in der Theorie aufgestellten Spannungs- bzw. Widerstandsverhältnisse. Dieses Vorgehen wird
für einige Schaltbilder umgesetzt. Die Amplitude wird während des ganzen Versuchs auf einen Wert von 0.5V gestellt.
Ab einem Volt könnte die Apparatur Schaden nehmen. \\ \noindent 
Zu Anfang wird der Wert eines Ohm'schen Widerstand mithilfe der 
Wheatonschen Brückenschaltung nach dem gerade beschriebenen Vorgehen bestimmt. Dies wird für einen
zweiten unbekannten Ohm'schen Widerstand durchgeführt.\\

Die nächste Schaltung ist eine Kapazitätsmessbrücke, mit dessen Gebrauch die Kapazität eines Kondensators
bestimmt wird. Auch hier gibt es wieder zwei Messungen, nach bereits erläutertem Vorgehen für zwei Kondensatoren.
In diesem Schaltbild wird ebenfalls ein RC-Glied verwendet.
\\
Nun soll die Induktivität einmal mittels einer Induktivitätsmessbrücke, und ein zweites Mal mithilfe der Maxwell-Brücke
bestimmt werden. Bis zu diesem Zeitpunkt wird die Frequenz auf einem festen Wert gelassen. 
\\
Zur letzten Messreihe wird die Frequenz nicht mehr konstant gelassen, sondern stufenweise erhöht, denn es soll die 
Frequenzabhängigkeit der Brückenspannung untersucht werden. Dazu wird hier die Wien-Robinson-Brücke 
verwendet. In einem Messbereich von $50$ bis $500$\,Hz wird jeweils in Schritten von $50$\,Hz die Amplitude der 
auf dem Oszilloskop angezeiten Schwingung gemessen. Ab $\qty{500}{\hertz}$ wird die Messung in Schritten von 
$\qty{500}{\hertz}$ fortgesetzt. 
