\section{Durchführung}
\label{sec:Durchführung}
Der Versuchsaufbau ist gegeben durch ein digitales Oszilloskop, einen Funktionengenerator, verschiedenartige 
Widerstände sowie Kabel. Ebenso war auch ein Tiefpass-Filter vorhanden, welcher aber nicht zwingend zur
Durchführung benötigt wurde. Hauptaufgabe ist es, die Kondensatoren, Spulen sowie Ohmschen Widerstände in 
nach verschiedenen Schaltbildern aus der \ref{sec:Theorie} aufzubauen. Damit werden dann jeweils Werte von den 
jeweils induktiven, Ohm'schen oder kapazitiven Widerstände ermittelt. Ist eine Schaltung aufgebaut, wird 
der verstellbare Widerstand so justiert, sodass die auf dem Oszilloskop dargestellte Schwingung ihr Amplituden-
Minimum erreicht. An diesem Punkt, an dem die Amplitude genähert gleich null ist, gelten 
die in der \ref{sec:Theorie} aufgestellten Spannungs- bzw. WIderstandsverhältnisse. Dieses Vorgehen wird
für einige Schaltbilder umgesetzt. Die Amplitude wird während des ganzen Versuchs auf einen Wert von 0.5V gestellt.
Ab einem Volt könnte die Apparatur Schaden nehmen Zu Anfang wird der Wert eines Ohm'schen Widerstand mithilfe der 
Wheatoneschen Brückenschaltung nach dem  gerade beschriebenen Vorgehen bestimmt. Dies wird für einen
zweiten unbekannten Ohm'schen Widerstand durchgeführt.\\
\\
Schaltbild??
\\
Die nächste Schaltung ist eine Kapazitätsmessbrücke, mit dessen Gebrauch die Kapazität eines Kondensators
bestimmt wird. Auch hier gibt es wieder zwei Messungen nach bereits erläutertem Vorgehen für zwei Kondensatoren.
In diesem Schaltbild wird ebenfalls ein RC-Glied verwendet.

\\
Nun soll die Induktivität mittels einer Induktivitätsmessbrücke einmal, und mithilfe der Maxwell-Brücke ein zweites Mal 
bestimmt werden. Bis zu diesem Zeitpunkt wird die Frequenz auf einem festen Wert gelassen. 

