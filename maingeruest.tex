\documentclass[titlepage=firstiscover, parskip=half, bibliography=totoc,captions=tableheading]{scrartcl} % KOMA-Script Dokumentenklasse Article

% Warnung, falls noch einmal kompiliert werden muss
\usepackage[aux]{rerunfilecheck}

% Paket für Schriftarteinstellung, muss immer geladen werden
\usepackage{fontspec}

% Deutsche Spracheinstellungen, wichtig z. B. für korrekte Trennung
\usepackage[ngerman]{babel}

% mehr Pakete hier
\usepackage{amsmath} % unverzichtbare Mathe-Befehle
\usepackage{amssymb} % viele Mathe-Symbole
\usepackage{mathtools} % Erweiterungen für amsmath
\usepackage[
math-style=ISO, % ┐
bold-style=ISO, % │
sans-style=italic, % │ ISO-Standard folgen
nabla=upright, % │
partial=upright, % │
mathrm=sym, % ┘
]{unicode-math}
\usepackage[
version=4,
math-greek=default,
text-greek=default,
]{mhchem}
\usepackage[section, below]{placeins}
\usepackage{caption} % Captions schöner machen
\usepackage{graphicx}
\usepackage{graphicx}
\usepackage{tabularray}
\usepackage[
locale=DE,
separate-uncertainty=true, % Immer Unsicherheit mit ±
per-mode=symbol-or-fraction, % m/s im Text, sonst \frac
% alternativ:
% per-mode=reciprocal, % m s^{-1}
% output-decimal-marker=., % . statt , für Dezimalzahlen
]{siunitx}

\UseTblrLibrary{booktabs}
\UseTblrLibrary{siunitx} % Lädt siunitx und definiert die S-Spalte
\usepackage[style=alphabetic]{biblatex} % nach babel
\addbibresource{lit.bib}

% Unterstützung für Links und PDF Metadaten
\usepackage[unicode]{hyperref}
\usepackage{bookmark}
\usepackage{microtype}
\usepackage{xfrac}
% Einstellungen hier, z.B. Fonts
\setmathfont{Latin Modern Math}
% \setmathfont{Tex Gyre Pagella Math} % alternativ


