\section{Zielsetzung}
\label{sec:Zielsetzung}
Ziel dieses Versuchs ist es eine Funktion in ihre Fourieranteile zu zerlegen und aus Fourieranteilen eine Funktion zusammenzufügen.

\section{Theorie}
\label{sec:Theorie}
\cite{sample}Fourier-Analysen werden in der Physik oft zur Bildverarbeitung verwendet.
Die Grundvorraussetzung hierfür sind periodische Funktionen, die zeitlich periodisch
\begin{align*}
    f(t + T) = f(t)
    \intertext{oder räumlich periodisch}
    f(x + D) = f(x)
\end{align*}
sein können.
Beispiele für periodische Funktionen sind Sinus- und Kosinusfunktionen.

Mit diesen beiden Funktionen können viele periodische Vorgänge in der Natur beschrieben werden.
Hierfür wird das Fouriersche Theorem verwendet, welches wie folgt aussieht:
\begin{align}
    \frac{1}{2} a_0 + \sum_{n = 1}^{\infty} (a_n \cos(\frac{2 \pi n}{T}t) + b_n \sin(\frac{2 \pi n}{T}t))
    \label{equ:Fourierreihe}
\end{align}
Wenn die Reihe gleichmäßig konvergiert, stellt sie eine periodische Funktion $f(t)$ mit der Periode $T$ dar.
Dies ist immer dann der Fall, wenn $f(t)$ stetig ist.
An Stellen, an denen $f$ unstetig ist, kann die Fourier-Reihe die Funktion nicht approximieren und es entsteht eine endliche Abweichung.
In Formel \ref{equ:Fourierreihe} kommen nur die Phasen $0, \pi/2 $ und $3\pi/2$ vor.
Außerdem treten nur ganzzahlige Vielfache der Grundfrequenz $\nu = \frac{1}{T}$ auf.
Als Grundfrequenz wird die Frequenz des periodischen Vorgangs beschrieben.
Die Koeffizienten $a_n$ und $b_n$ werden wie folgt berechnet
\begin{align}
    a_n = \frac{2}{T} \int_0^T f(t) \cos(\frac{2 \pi n}{T}t) \, \symup{d}t \ , n = 1, 2, ... \\
    \intertext{und}
    b_n = \frac{2}{T} \int_0^T f(t) \sin(\frac{2 \pi n}{T}t) \, \symup{d}t \
    , n = 1, 2, ... \, .
\end{align}

Für gerade Funktionen gilt $f(t) = f(-t)$ und somit verschwinden alle Koeffizienten $b_n$.
Bei ungerade Funktionen verschwinden alle $a_n$, da $f(t) = -f(-t)$ gilt.

Mit der Fourier-Transformation
\begin{equation}
    g(\omega) = \frac{1}{2\pi} \int_{-\infty}^{\infty} f(t) e^{i\omega t} \, \symup{d}t
\end{equation}
kann das gesamte Frequenzspektrum einer Funktion bestimmt werden.
Das Frequenzspektrum einer nicht-periodischen Funktion ist kontinuierlich, während das einer periodischen Funktion aus einer Reihe von $\delta$-Funktionen besteht.
Die Transformation lässt sich auch wieder umkehren mit
\begin{equation}
    f(t) = \frac{1}{2\pi} \int_{-\infty}^{\infty} g(\omega) e^{-i\omega t} \, \symup{d}\omega \, .
\end{equation}
Da oft nicht über unendliche Zeiträume integriert werden kann, wird die Integration auf einen endlich langen Zeitraum beschränkt.