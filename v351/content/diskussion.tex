\section{Diskussion}
\label{sec:Diskussion}
Der zur Erzeugung verschiedener Oberwellen genutzte Oberwellengenerator stellte sich als fehlerhaft heraus.
Die Amplitude der ersten Oberwelle änderte sich mit an- und abschalten des Schalters \"Summation\" an der ersten Obewelle.
Der Schalter wurde daraufhin für die Dauer der Durchführung auf \"ein\" gelassen, da dort einereseits die höhere Amplitude
zu verzeichnen war, andererseits wird die erste Oberwelle immer gebraucht.\\  \noindent
Dieser Fehler hat jedoch zu keinem Zeitpunkt die Durchführung eingeschränkt oder verkompliziert. Durch leichtes Rauschen 
der Apparatur konnten bei der Synthese die Amplituden der letzten Oberwellen nur schwierig bis garnicht einstellen, vor allem
stellte sich dieses Problem bei der Dreieckspannung durch die $\sfrac{1}{n^2}$-Abhängigkeit der Amplitude dar. Die Synthese konnte 
trotzdem gut durchgeführt werden, da die größerzahligen Oberwellen einen immer kleineren Beitrag zum Gesamtbild leisten. Die Ergebnis-Bilder
der Synthese sahen also so aus wie erwartet. \\ \noindent
Das schon erwähnte Rauschen machte das Messen hier im Berech kleiner Amplituden wieder schwierig bis unmöglich. Daher wurde die Messung der Peaks
bei der Analyse abgebrochen, sobald das Rauschen zu groß wurde. Bis dorthin wurden jedoch sehr sinnvolle Messwerte abgelesen, die jeweils
sehr nah am Theoriewert, jedoch leicht außerhalb der von Python errechneten Fehlertoleranz liegen. Die Abweichung beträgt zwei mal 
$0,3\%$, ein mal liegt sie bei $0,4\%$. Bei der größeren Abweichung ist der errechnete Fehler jedoch auch leicht erhöht. Abschließend lässt sich also
sagen, dass die Synthese sowie die Analyse gute, zu erwartende Ergebnisse geliefert haben.


