\section{Auswertung}
\label{sec:Auswertung}
    In \autoref{tab:Messdaten} sind beide gemessenen Temperaturen,
    Drücke sowie die Arbeit in Abhängigkeit der Zeit aufgetragen.
    Die Messunsicherheiten betragen $\pm 0.1\,°C$ für beide Temperaturen,
    $\pm 0.5\,°C$ für beide Drücke sowie $\pm 5$\, Watt für die Arbeit.
    \begin{table}[H]
      \centering
      \label{tab:Messdaten}
      \caption{Die Messwerte beider Drücke, der Temperaturen beider Reservoirs sowie 
      die Arbeit zu verschiedenen Zeiten.}
      \begin{tblr}{colspec={c c c c c c}}
      \toprule
      $t$ / min & $T_\text{b}$ / °C & $p_\text{b}$ / bar &
      $T_\text{a}$ / °C & $p_\text{a}$ / bar & $A$ / W\\ 
      \midrule
      1  & 21,5 & 4,4  &  21,0 & 4,1 & 150\\
      2  & 22,0 & 6,0  &  21,0 & 1,4 & 165\\
      3  & 22,8 & 6,2  &  21,0 & 1,6 & 175\\
      4  & 23,8 & 6,6  &  20,4 & 2,0 & 185\\
      5  & 26,7 & 7,5  &  17,4 & 2,1 & 200\\
      6  & 28,5 & 8,0  &  15,6 & 2,2 & 200\\
      7  & 30,3 & 8,2  &  13,9 & 2,2 & 200\\
      8  & 32,0 & 8,6  &  12,2 & 2,2 & 205\\
      9  & 33,7 & 9,0  &  10,6 & 2,2 & 205\\
      10 & 35,4 & 9,4  &  9,0  & 2,2 & 206\\
      11 & 37,0 & 9,7  &  7,5  & 2,2 & 210\\
      12 & 38,5 & 10,1 &  6,2 & 2,3 & 210\\
      13 & 40,1 & 10,4 &  5,1 & 2,3 & 210\\
      14 & 41,6 & 10,8 &  4,1 & 2,3 & 210\\
      15 & 43,0 & 11,0 &  3,4 & 2,3 & 210\\
      16 & 44,4 & 11,5 &  2,8 & 2,3 & 210\\
      17 & 45,6 & 11,7 &  2,2 & 2,3 & 210\\
      18 & 46,8 & 12,0 &  1,7 & 2,3 & 210\\
      19 & 48,0 & 12,2 &  1,3 & 2,3 & 210\\
      20 & 49,0 & 12,5 &  0,9 & 2,2 & 205\\
      21 & 50,0 & 12,8 &  0,4 & 2,2 & 205\\
      \bottomrule
      \end{tblr}
    \end{table}
    \begin{figure}
        \centering
        \includegraphics[height= 5cm]{build/plota.pdf}
        \caption{Die Temperatur in Abhängigkeit der Zeit der beiden Reservoire.}
        \label{fig:Fit_Temperatur}
    \end{figure}
    \subsection{Ausgleichsrechnung}
    \autoref{fig:Fit_Temperatur} zeigt die Messwerte der Messreihen für $T_1$
    und $T_2$ sowie die Plots einer nichtlinearen Ausgleichsrechnung der Messwerte.
    Für die Ausgleichsrechnung werde die Funktion
    \begin{equation}
        T(t) = at^2+bt+c
        \label{eq:Ausgleich}
    \end{equation}
    gewählt.
    Mittels Python werden folgende Werte errechnet
    \begin{align*}
        a_1 &= \SI{-4.6(1.1)e-6}{\degreeCelsius\per\square\second}
        & a_2 &= \SI{10.4(1.9)e-6}{\degreeCelsius\per\square\second}\\
        b_1 &= \SI{3.2(0.2)e-2}{\degreeCelsius\per\second}
        & b_2 &=\SI{3.3(0.3)e-2}{\degreeCelsius\per\second}\\
        c_1 &=\SI{18.0(0.4)}{\degreeCelsius}
        & c_2 &=\SI{25.6(0.7)}{\degreeCelsius}
    \end{align*}
    wobei die Werte mit dem Index 1 den Verlauf der blauen Kurve beschreiben
    und die Werte mit dem Index 2 den der orangenen Kurve beschreiben.

    \subsection{Berechnung der Differentialquotienten}
    Die Differentialquotienen $\sfrac{\text{d}T_i}{\text{d}t}$ berechnen sich durch einsetzen von $t$ in die Ableitung
    von \autoref{eq:Ausgleich}, $T'(t) = 2at+b$\,.
    Die errechneten Werte der Ableitung sind in \autoref{tab:Diffquot} dargestellt.
    \begin{table}[H]
      \centering
      \begin{tabular}{
        S[table-format=4.0]
        S[table-format=2.3]
        S[table-format=2.3]
        S[table-format=3.0]
      }
        \toprule
        {$t\left[\unit{s}\right]$} & {$\frac{\text{d}T_1}{\text{d}t}\left[\unit{\frac{°C}{s}}\right]$}
        & {$\frac{\text{d}T_2}{\text{d}t}\left[\unit{\frac{°C}{s}}\right]$} & {$N\,\left[\unit{W}\right]$}\\
        \midrule
        300 &  {$0,029 \pm 0,002$}  & {$-0,027 \pm 0,004$} & 200\\
        600 &  {$0,026 \pm 0,003$}  & {$-0,021 \pm 0,005$} & 206\\
        900 &  {$0,023 \pm 0,004$}  & {$-0,015 \pm 0,006$} & 210\\
        1200 & {$0,021 \pm 0,004$}  & {$-0,008 \pm 0,007$} & 205\\
        \bottomrule
    \end{tabular}
      \label{tab:Diffquot}
    \caption{Differentialquotienten für vier Zeiten der zwei Temperaturverläufe.}
    \end{table}

    \subsection{Berechnung der Güteziffer}
    Die ideale Güterziffer lässt sich über \autoref{eq:Gueteideal} berechnen.
    Die reale Güteziffer wird mit \autoref{eq:Guetereal} berechnent.
    Hierbei gilt $m_1 = 3\,\unit{kg}$ und $c_w = \SI{4200}{\joule\per\kg\per\kelvin}$.
    Die Wärmekapazität der Kupferschlange und der Reservoire ist $m_\text{k} c_\text{k} = \SI{750}{\joule\per\kelvin}$.
    Die Güteziffern sind in \autoref{tab:Guete} aufgeführt.
    \begin{table}
  \centering
  \begin{tabular}{
    S[table-format=4.0]
    S[table-format=2.3]
    S[table-format=2.3]
    S[table-format=2.3]
  }
    \toprule
    {$t\left[\unit{s}\right]$} & {$v_{\text{ideal}}$} & {$v_{\text{real}}$ für $T_1$} & {$\Delta v_1$}\\
    \midrule
    300 & 22.38& {$1,9 \pm 0,1$} & 0.94 \\
    600 & 9.51 & {$1,7 \pm 0,2$} & 0.34 \\
    900 & 6.63 & {$1,5 \pm 0,3$} & 0.37 \\
    1200& 5.52 & {$1,4 \pm 0,3$} & 0.35 \\
    \bottomrule
\end{tabular}
\caption{Ideale und reale Güteziffer für vier Zeiten und deren Abweichung.}
        \label{tab:Guete}
\end{table}

    \subsection{Die Verdampfungswärme L}
    Die Verdampfungswärme des Transportgases kann aus seiner Dampfdruckkurve bestimmt werden.
    Die Dampfdruck-Kurve wird aus den Messwerten $T_1$ und $p_\text{b}$ gewonnen.
    Dafür wird der Logarithmus der Drucks $p_\text{b}$ gegen die reziproke absolute Temperatur $T_1$ aufgetragen.
    \begin{figure}[H]
        \centering
        \includegraphics[height=7cm]{build/Verdampfungswärme.pdf}
        \caption{Die Messwerte des warmen Reservoirs aufgetragen als
        der Logarithmus der Drucks $p_\text{b}$ gegen die reziproke absolute Temperatur
        $T_1$ mit der Ausgleichsgeraden. $p_0 = \qty{1001}{\milli\bar}$}
    \end{figure}
    Die Gleichung zur Bestimmung von $L$ und dem damit einhergehenden Fit ist
    \begin{equation}
        \ln(p) = - \frac{L}{R} \cdot \frac{1}{T}
        \Rightarrow y = m \cdot x + b \, \text{.}
    \end{equation}
    Nummeric Python berechnet $m = \SI{2726 \pm 179}{\kelvin}$ und $b = 4.1 \pm 0.6$.
    Damit ergibt sich mit $L = -a \cdot R$, wobei $R$ die Universelle Gaskonstante mit dem Wert
    $R = \SI{8.314}{\joule\per\mole\per\kelvin}$ \cite{Gaskonstante},
    \begin{equation*}
        L = \SI{2.27(0.15)e+4}{\joule\per\mol} \, \text{.}
    \end{equation*}
\subsection{Massendurchsatz}
Der Massendurchsatz wird durch Umstellen der \autoref{eq:massendurchsatz} errechnet.
 \begin{table}
  \centering
  \begin{tabular}{
    S[table-format=4.0]
    S[table-format=2.3]
  }
    \toprule
    {$\frac{\Delta m}{\Delta t}(1)$} & {$-1.9 \pm 0.3 \, \unit{\frac{g}{s}}$} \\
    \addlinespace
    {$\frac{\Delta m}{\Delta t}(2)$} & {$-1.5 \pm 0.4 \, \unit{\frac{g}{s}}$} \\
    \addlinespace
    {$\frac{\Delta m}{\Delta t}(3)$} & {$-1.0 \pm 0.4 \, \unit{\frac{g}{s}}$} \\
    \addlinespace
    {$\frac{\Delta m}{\Delta t}(4)$} & {$-0.6 \pm 0.5 \, \unit{\frac{g}{s}}$} \\
    \bottomrule
\end{tabular}
\caption{Die Massendurchsätze und deren Abweichungen der vier Zeiten.}
        \label{tab:huhu}
\end{table}
Die Werte sind in \autoref{tab:huhu} dargestellt.
\subsection{Bestimmung der mechanischen Arbeit}
Der Sinn einer Wärmepumpe ist, im idealen Falle nur die Arbeit
zur Phasenumwandlung aufzuwenden. Diese mechanisch geleistete Arbeit soll nun berechnet
werden. Um die dazu in der Theorie angegebene \autoref{eq:mechanisch} anwenden zu können,
muss zunächst $\rho$ bestimmt werden. Dazu wird die allgemeine Gasgleichung verwendet
\begin{equation}
  pV=n\text{R}T\,.
\end{equation}
Es gilt $n\text{R}=\text{const.}$, woraus folgt, dass $n_1\text{R}=n_2\text{R}$ korrekt ist.
Über dies und die Beziehung $V=\sfrac{m}{\rho}$ lässt sich für $\rho$ herleiten, dass
\begin{equation}
  \frac{p_0V_0}{T_0}=\frac{p_2V_2}{T_2}\Leftrightarrow
  \frac{p_0}{\rho_0T_0}=\frac{p_\text{a}}{T_2\rho}\Leftrightarrow
  \rho=\frac{\rho_0T_0p_\text{a}}{T_2p_0}\,.
\end{equation}
Die Einheit $\unit{\bar}$ sowie $\unit{\gram\per\second}$ müssen jeweils in SI-EInheiten 
umgerechnet werden. Um mittels Formel \eqref{eq:mechanisch} nun die mechanische 
Arbeit berechnen zu können, sind einige Werte aus der Aufgabenstellung gegeben.
Für das Transportgas $\text{Cl}_2\text{F}_2\text{C}$ gilt die Dichte 
$\rho_0=\qty{5.51}{\gram\per\liter}$ bei $T_0=\qty{0}{\celsius}$, $p=\qty{1}{\bar}$
sowie $k=1,14$\,. Folgend ergeben sich für die Kompressorleistung sowie die Dichte
die in Tabelle \ref{tab:leistung} eingetragenen Werte.
\begin{table}
  \centering
  \label{tab:leistung}
  \caption{Die Dichte $\rho$ und die darüber berechnete Kompressorleistung zu vier 
  verschiedenen Messzeiten.}
  \begin{tblr}{colspec={c c c}}
    \toprule
    $t$ / s & $\rho$ / $\unit{\kilogram\per\liter}$ & Leistung / W \\
    \midrule
    300 & 5,2 & 95 $\pm$ 6,0\\
    600 & 5,3 & 86 $\pm$ 6,0\\
    900 & 5,4 & 67 $\pm$ 4,0\\
    1200& 5,5 & 41 $\pm$ 2,7\\
    \bottomrule
  \end{tblr}
\end{table}