\section{Diskussion}
\label{sec:Diskussion}
Für die Abweichungen der idealen zur realen Güteziffer ergeben
sich für die 4 verschiedenen Zeiten von 5 bis 20 Minuten Abweichungen von $92\,\%$,
$82\%$, $77\%$ sowie $74\%$. Die festgestellten Abweichungen befinden sich deutlich 
außerhalb des berechneten Fehlers. Dies lässt sich vor allem auf die sehr 
schlechte Wärmedämmung der Apparatur zurückführen. Die Deckel der
Reservoirs saßen nicht ganz dicht beziehungsweises luftdicht auf, da sie nur darüber 
geschoben werden konnten. Ebenso ist die Isolierung der gesammten Apparatur
nicht optimal. Da 5 Messwerte theoretisch gleichzeitig abgenommen werden mussten,
bringt der Zeitversatz des zyklischen Ablesens eine Unsicherheit in die Daten,
dessen Größenordnung sich nich abschätzen lässt. Nach einigen Messungen hat die 
Routine bewirkt, dass das Ablesen der Werte schneller von Statten ging. Dies hat 
eine kleine zusätzliche Verzerrung der zeitlichen Skala herbeigeführt. 
