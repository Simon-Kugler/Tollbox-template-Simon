\section{Auswertung}
\label{sec:Auswertung}

\subsection{Druck unter einem Bar}
Mit den im Anhang zu findenden Messwerten lassen sich folgende Grafiken erstellen:
\begin{figure}[H]
  \begin{subfigure}{0.48\textwidth}
      \includegraphics[height=6cm]{plota.pdf}  
    \caption{Bereich von $30$ bis $\SI{1000}{\milli\bar}$. $p_0=\SI{1}{\milli\bar}$.}
    \label{fig:MesswerteKlein}
  \end{subfigure}
  \hfill
  \begin{subfigure}{0.48\textwidth}
    \includegraphics[height=6cm]{plotb.pdf}
    \caption{Bereich von $1$ bis $\SI{15}{\bar}$. $p_0=\SI{1}{\bar}$}
    \label{fig:MesswerteGross}
  \end{subfigure}
  \caption{Die Messwerte der ersten Messreihe aufgetragen als der Logarithmus des Drucks $p$
  gegen die reziproke absolute Temperatur $T$.}
  \label{fig:Teila}
\end{figure}
Um daraus die zu berechnende Verdampfungswärme des Wassers zu bestimmen, wird eine Gerade durch die Messwerte gelegt.
Diese wird mittels Python für den Bereich von $30$ bis $1000$\,mbar erstellt. \\
\begin{figure}[H]
  \centering
  \includegraphics[scale=0.5]{plotc.pdf}
  \caption{Die Messwerte der zweiten Messreihe aufgetragen als der Logarithmus des Drucks $p$
  gegen die reziproke absolute Temperatur $T$ mit der Ausgleichsgerade. $p_0=1$\,mbar.}
  \label{fig:Ausgleichsgerade}
\end{figure}
\newpage
Die Gleichung zur Bestimmung von $L$ und dem damit einhergehenden Fit ist wie folgt:
\begin{equation}
  ln(p) = - \frac{L}{R} \cdot \frac{1}{T}
  \Rightarrow y = a \cdot x + b = -2029 \cdot x + 17.010
\end{equation}
Mit numeric Python ergeben sich folgende Messunsicherheiten: $a = \SI{-2029 \pm 21 }{\frac{1}{\kelvin}}$
und $b = \SI{17.010 \pm 0.063}{\frac{1}{\kelvin}}$.
Die Verdampfungswärme wird mit folgendem Wert berechnet:
\begin{equation*}
  L = - \ a \cdot R \Rightarrow L = \SI{1.88 +- 0.09e4}{\frac{\J}{\mol}}
\end{equation*}
L ist die Steigung der Ausgleichsgeraden in Abbildung \ref{fig:Ausgleichsgerade} multipliziert mit der Universellen Gaskonstante R.
\noindent
Jetzt soll die äußere Verdampfungswärme $L_a bestimmt werden$.
Diese beschreibt die benötigte Arbeit, um bei konstantem Druck das Volumen eines Stoffs zu verändern.
Hierfür wird die ideale Gasgleichung mit der verrichteten Arbeit gleich gesetzt.
\begin{equation}
    P \cdot V = R \cdot T = L_a
\end{equation}
Also ergibt sich $L_a = \SI{3101.3}{\frac{\J}{\mol}}$.
Um die erforderliche Arbeit zur Überwindung der molekularen Anziehungskraft bei Verdampfung $L_i$ zu bestimmen, wird
die Differenz zwischen $L$ und $L_a$ gebildet.
\begin{equation}
    L_i = L - L_a \Rightarrow L_i = \SI{13768.8 \pm 174.6}{\frac{\J}{\mol}}
\end{equation}

\subsection{Druck über einem Bar}
Auflösen der Clausius-Clapeyronschen Gleichung nach $L$ zur Bestimmung der Wärmeabhängigkeit ergibt.\\
\begin{align}
  &(V_D-V_F)dp=\frac{L}{T}dT\nonumber\\
  \Leftrightarrow L=&(V_D-V_F)\frac{dp}{dT}T\label{eqn:L(V,T)}
\end{align}
Mittels Python und scipy wird polynomialer Fit dritten Grades errechnet. Die 
zum Polynom $p(T)=a\cdot T^3+b\cdot T^2+c\cdot T+d$ gehörenen Vorfaktoren sind:
\begin{align*}
  a &= (0.021 ± 0.006)\si{\pascal\per\kelvin\cubed}\\
  b &= (-24.7 ± 7.7)\si{\pascal\per\kelvin\squared}\\
  c &= (9879 ± 3300)\si{\pascal\per\kelvin}\\
  d &= \num{-1.33(0.47)e6}\si{\pascal}\\
\end{align*}
\begin{figure}[h]
\centering
\includegraphics[height=7cm]{plotd.pdf}
\caption{Druck und Temperatur der zweiten Messreihe, $p\geq 1$Bar, sowie 
der Fit durch die Messwerte.}
\label{fig:Druck_groß}
\end{figure}
\\
Ableitung des Polynoms $p(T)$ ergibt
\begin{equation}
p'(T)=3\cdot a\cdot T^2+2\cdot b\cdot T+c.
\end{equation}
Das ist der Asudruck für $\frac{dp}{dT}$.\\
In der Theorie wurde unter einer Bedingung die Annahme getroffen, dass $V_F$ gegen $V_D$
vernachlässigt werden darf. Dies wird hier genutzt. Nach umstellen der Formel aus der Theorie ergibt sich für $V=V_D$:
\begin{align}
   RT&=\left(p+\frac{a}{V^2}\right)V\\
   a\text{ ist hierbei } \text{gleich} &0,9\,\text{J m}^3\text{mol}^{-2}.\nonumber 
   \intertext{ist. Mithilfe der pq-Formel ergibt sich}
   \Rightarrow V&=\frac{RT}{2p}\pm\sqrt{\left(\frac{RT}{2p}\right)^2+\frac{a}{p}}\label{eqn:V(T)}
   \intertext{Durch einsetzen von Gl.\eqref{eqn:V(T)} in Gl.\eqref{eqn:L(V,T)} lässt sich folgender Ausdruck für L finden:}
   L(T)&=\frac{T}{P}\left(\frac{RT}{2}\pm\sqrt{\left(\frac{R^2T^2}{4}\right)-ap}\right).
\end{align}
\newpage
Mit Python geplottet sieht $L(T)$ wie folgt aus:
\begin{figure}
  \begin{subfigure}{0.45\textwidth}
  \centering
  \includegraphics[height=4cm]{plote.pdf}
  \caption{$L$ in Abhängigkeit von T für $V_{D+}$.}
  \label{fig:Verdampfungswärme1}
  \end{subfigure}
  \hfill
  \begin{subfigure}{0.45\textwidth}
  \centering
  \includegraphics[height=4cm]{plotf.pdf}
  \caption{$L$ in Abhängigkeit von T für $V_{D-}$.}
  \label{fig:Verdampfungswärme2}
  \end{subfigure}
\end{figure}
