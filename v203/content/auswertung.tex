\newpage
\section{Auswertung}
\label{sec:Auswertung}

\begin{table}[H]
  \centering
  \caption{Temperatur und Druck bei Verdampfung des Wassers. Der Druck hat eine Messunsicherheit von
  $\pm$mB, die Temperatur von $\pm 1$ K.}
  \label{tab:Messreihe_1}
\begin{tabular}{
  c c||c c||c c||c c
}
\toprule 
$T/ \unit{\kelvin}$ & $p / \text{mB}$ & $T/ \unit{\kelvin}$ & $p / \text{mB}$&
$T/ \unit{\kelvin}$ & $p / \text{mB}$ & $T/ \unit{\kelvin}$ & $p / \text{mB}$ \\
\midrule
293.15  & 40   & 313.15  & 372  & 333.15  & 528  & 353.15  & 773 \\
294.15  & 167  & 314.15  & 379  & 334.15  & 537  & 354.15  & 790 \\
295.15  & 228  & 315.15  & 385  & 335.15  & 549  & 355.15  & 810 \\
296.15  & 255  & 316.15  & 395  & 336.15  & 562  & 356.15  & 826 \\
297.15  & 274  & 317.15  & 400  & 337.15  & 571  & 357.15  & 844 \\
298.15  & 287  & 318.15  & 405  & 338.15  & 578  & 358.15  & 856 \\
299.15  & 296  & 319.15  & 412  & 339.15  & 586  & 359.15  & 879 \\
300.15  & 303  & 320.15  & 421  & 340.15  & 600  & 360.15  & 901 \\
301.15  & 310  & 321.15  & 431  & 341.15  & \text{--} .& 361,15  & 913 \\
302.15  & 316  & 322.15  & 439  & 342.15  & 628  & 362.15  & 933 \\
303.15  & 322  & 323.15  & 445  & 343.15  & 638  & 363.15  & 944 \\
304.15  & 327  & 324.15  & 453  & 344.15  & 650  & 364.15  & 966 \\
305.15  & 331  & 325.15  & 461  & 345.15  & 660  & 365.15  & 979 \\
306.15  & 336  & 326.15  & 465  & 346.15  & 670  & 366.15  & 990 \\
307.15  & 341  & 327.15  & 476  & 347.15  & 690  & 367.15  & 1015\\
308.15  & 347  & 328.15  & 482  & 348.15  & 701  & 368.15  & 1030\\
309.15  & 351  & 329.15  & 491  & 349.15  & 716  & 369.15  & 1046\\
310.15  & 356  & 330.15  & 500  & 350.15  & 723  & 370.15  & 1061\\
311.15  & 362  & 331.15  & 511  & 351.15  & 743  & 371.15  & 1079\\
312.15  & 367  & 332.15  & 520  & 352.15  & 760  & 372.15  & 1093\\
      &   &       &       &       &      & 373.15  & 1108 \\
\bottomrule
\end{tabular}
\end{table}
\begin{table}[H]
  \centering
  \caption{Temperatur und Druck bei Verdampfung des Wassers für $p\geq 1$.Der Druck hat eine Messunsicherheit von
  $\pm$mB, die Temperatur von $\pm 1$ K.}
  \label{tab:Messreihe_2}
\begin{tabular}{
  c c||c c
}
\toprule 
$p$/B & $T$/K & $p$/B & $T$/K\\
\midrule
1000  & 391.15 & 9000  & 447.15\\
2000  & 404.15 & 10000 & 451.15\\
3000  & 413.15 & 11000 & 454.15\\
4000  & 419.15 & 12000 & 459.15\\
5000  & 427.15 & 13000 & 461.15\\
6000  & 433.15 & 14000 & 463.15\\
7000  & 438.15 & 15000 & 465.15\\
8000  & 443.15 &       &      \\
\bottomrule
\end{tabular}
\end{table}
\subsection{Druck unter einem Bar}
Mit diesen Messwerten lassen sich folgende Grafiken erstellen:
\begin{figure}[H]
  \begin{subfigure}{0.48\textwidth}
      \includegraphics[height=6cm]{python/plota.pdf}
    \caption{Bereich von $30$ bis $1000 mbar$}
    \label{fig:MesswerteKlein}
  \end{subfigure}
  \hfill
  \begin{subfigure}{0.48\textwidth}
    \includegraphics[height=6cm]{python/plotb.pdf}
    \caption{Bereich von $1$ bis $15 bar$}
    \label{fig:MesswerteGross}
  \end{subfigure}
  \caption{Die Messwerte der ersten Messreihe aufgetragen als der Logarithmus des Drucks $p$
  gegen die reziproke absolute Temperatur $T$.}
  \label{fig:Teila}
\end{figure}
Um hier raus die Verdampfungswärme von Wasser zu bestimmen, legt man eine Gerade durch die Messwerte.
Diese wird mittels Python für den Bereich von $30$ bis $1000 mbar$ erstellt und sieht folgendermaßen aus: \\
\begin{figure}[H]
  \centering
  \includegraphics[scale=0.5]{python/plotc.pdf}
  \caption{Die Messwerte der zweiten Messreihe aufgetragen als der Logarithmus des Drucks $p$
  gegen die reziproke absolute Temperatur $T$ mit der Ausgleichsgerade.}
  \label{fig:Ausgleichsgerade}
\end{figure}
Für diesen Fit wurden die ersten zwei Messwerte dieser Reihe nicht berücksichtigt, da diese die Bestimmung von $L$
offensichtlich verfälschen würden.
Die Gleichung des zu der Ausgleichsgerade ist:
\begin{equation}
  ln(p) = - \frac{L}{R} \cdot \frac{1}{T}
  \Rightarrow y = a \cdot x + b = -2029 * x + 17.010
\end{equation}
Mit Numerik Python ergeben sich folgende Messunsicherheiten: $a = \SI{-2029 \pm 21 }{\frac{1}{\kelvin}}$
und $b = \SI{17.010 \pm 0.063}{\frac{1}{\kelvin}}$
Für die Verdampfungswärme von Wasser ergibt sich zu
\begin{equation*}
  L = - \ a \cdot R \Rightarrow L = \SI{16870.1 \pm 174.6}{\frac{\J}{\mol}}
\end{equation*}
L ist die Steigung der Ausgleichsgeraden in Abbildung \ref{fig:Ausgleichsgerade} multipliziert mit der Universellen Gaskonstante R.

Nun folgt die Bestimmung der äußeren Verdampfungswärme $L_a$.
Diese beschreibt die benötigte Arbeit, um bei konstantem Druck das Volumen eines Stoffs zu verändern.
Hierfür wird die ideale Gasgleichung mit der verrichteten Arbeit gleich gesetzt.
\begin{equation}
    W = P \cdot V = R \cdot T = L_a
\end{equation}
Also ergibt sich $L_a = \SI{3101.3}{\frac{\J}{\mol}}$.
Um jetzt noch die erforderliche Arbeit zu Überwindung der molekularen Anziehungskraft bei Verdampfung $L_i$ zu bestimmen, wird
    die Differenz zwischen $L$ und $L_a$ gebildet.
\begin{equation}
    L_i = L - L_a \Rightarrow L_i = \SI{13768.8 \pm 174.6}{\frac{\J}{\mol}}
\end{equation}

\subsection{Druck über einem Bar}
Auflösen er Clausius-Clapeyronschen Gleichung nach $L$ zur Bestimmung der Wärmeabhängigkeit.\\
\begin{align}
  &(V_D-V_F)dp=\frac{L}{T}dT\nonumber\\
  \Leftrightarrow L=&(V_D-V_F)\frac{dp}{dT}T
\end{align}
Errechnet man mittels Python und scipy einen polynomialen Fit dritten Grades, erhält das Polynom $ax^3+bx^2+cx+d$ folgende Werte:
\begin{align*}
  a = 0.020709167025828 ± 0.005971965168968\\
  b = -24.652993974197063 ± 7.691549806645858\\
  c = 9879.126807499247661 ± 3297.004781181593444\\
  d = -1330791.268963911104947 ± 470353.071982218127232\\
\end{align*}
\begin{figure}[h]
\centering
\includegraphics[height=7cm]{Python/plotd.pdf}
\caption{Druck und Temperatur der zweiten Messreihe, $p\geq 1$Bar.}
\label{fig:Druck_groß}
\end{figure}
\\
Ableitung des Polynoms
\begin{equation}
p'(T)=3\cdot a\cdot T^2+2\cdot b\cdot T+c
\end{equation}
Das ist nun unser Asudruck für $\frac{dp}{dT}$.\\
Nun wird die, oben in der Theorie, schon aufgestellte Annahme genutzt, dass $V_F$ gegen $V_D$
vernachlässigt werden darf. Nach umstellen der Formel aus der Theorie ergibt sich für $V_D$:
\begin{align}
   RT&=\left(p+\frac{a}{V^2}\right)V
   \intertext{wobei }
   a&=0,9\,\frac{\text{Joule m}^3}{\text{Mol}^2}\nonumber 
   \intertext{ist. Mithilfe der pq-Formel ergibt sich}
   \Rightarrow V&=\frac{RT}{2p}\pm\sqrt{\left(\frac{RT}{2p}\right)^2+\frac{a}{p}}
   \intertext{Durch einsetzen von Gl.(7) in Gl.(5) lässt sich folgender Ausdruck für L schreiben:}
   L(T)&=\frac{T}{P}\left(\frac{RT}{2}\pm\sqrt{\left(\frac{R^2T^2}{4}\right)-ap}\right)
\end{align}
Mit Python geplottet sieht $L(T)$ wie folgt aus:
\begin{figure}
  \begin{subfigure}{0.3\textwidth}
  \centering
  \includegraphics[height=4cm]{Python/plote.pdf}
  \caption{$L$ in Abhängigkeit von T für $V_{D+}$.}
  \label{fig:Verdampfungswärme1}
  \end{subfigure}
  \hfill
  \begin{subfigure}{0.3\textwidth}
  \centering
  \includegraphics[height=4cm]{Python/plotf.pdf}
  \caption{$L$ in Abhängigkeit von T für $V_{D-}$.}
  \label{fig:Verdampfungswärme2}
  \end{subfigure}
\end{figure}
