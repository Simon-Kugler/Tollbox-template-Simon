\section{Diskussion}
\label{sec:Diskussion}

Bei der Messung wurde ein Wert nicht abgenommen bzw. aufgeschrieben. Dieser wurde in der Tabelle sowie in der Berechnung ausgelassen. 
Da er aber mittendrin lag und bei ihm keine große Abweichung zu erearten gewesen wäre, sollte dies keine große ABweichung in der 
Berechnung ausgelöst haben.\\
Da nach dem (bzw. davor) abstellen der Evakuierung die Heizung zu spät eingeschaltet wurde, erhöhte sich der Druck am Anfang durch 
minimale Undichtigkeiten in der Apparatur schon, obwohl noch kein signifikantes Ansteigen der Temepratur zu beobachten war. WIe in 
der Auswertung beschrieben und auch eindeutig zu erkennen, wurde diese beiden Messerte aus der Berechnung von $L$ ausgelassen.\\
$L$ hat einen Literaturwert von $\SI{40 660}{\frac{\joule}{\mol}}$. Wir kommen auf einen Wert von
$\SI{16 870,1\pm174,6}{\frac{\joule}{\mol}}$. Damit liegt der Literaturwert deutlich außerhalb unserer Fehlertoleranz. Dies ist einsersers evtl. auf der Weglassen der zwei Messwerte 
zurückzuführen. Außerdem war der Startdruck trotzdem noch deutlich über einem optimalen Vakuum, was die Messung auch deitlich verfälschen könnte. Außerdem war an diesem Tag ein leicht unter dem
Normaldruck liegenden Atmosphärendruck festzustellen, was aber wahrscheinlich keinen ernst zunehmenden Einfluss genommen haben dürfte. Als letzter Fehler lässt sich das evtl. nicht genaue
ablesen am analogen Thermometer ebenfalls als ein systematischer Fehler einstufen lässt.
