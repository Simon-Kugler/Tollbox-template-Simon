\section{Diskussion}
\label{sec:Diskussion}

Bei der Messung wurde ein Wert nicht aufgenommen. Dieser wurde in der Tabelle sowie in der Berechnung ausgelassen. 
Da er aber inmitten der anderen Werte lag und bei ihm keine große Abweichung zu erwarten gewesen wäre, sollte dies keine große Abweichung in der 
Berechnung ausgelöst haben.\\
Da nach dem (bzw. vor dem) abstellen der Evakuierung die Heizung zu spät eingeschaltet wurde, erhöhte sich der Druck am Anfang durch 
minimale Undichtigkeiten in der Apparatur schon, obwohl noch kein signifikantes Ansteigen der Temperatur zu beobachten war.\\ \noindent
$L$ hat einen Literaturwert von $\SI{40 660}{\frac{\joule}{\mol}}$\cite{Gaskonstante}.
Durch unsere Messung lässt sich ein Wert von$\SI{1.69(0.02)e4}{\joule\per\mol}$ ermitteln.
Der Literaturwert ist $140\%$ größer als dieser und damit weit außerhalb der Fehlertoleranz. \noindent
Der Startdruck war noch deutlich über einem optimalen Vakuum, was die Messung  verfälschen könnte.
Außerdem war an diesem Tag ein leicht unter dem Normaldruck liegender Atmosphärendruck festzustellen, was aber
wahrscheinlich keinen ernst zunehmenden Einfluss genommen haben dürfte.
Ebenso lässt sich das eventuell nicht genaue
Ablesen am analogen Thermometer ebenfalls als ein systematischer Fehler einstufen. Die jedoch größte Fehlerquelle wird
das rausnehmen der ersten beiden Messwerte sein.